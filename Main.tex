%% LyX 2.1.4 created this file.  For more info, see http://www.lyx.org/.
%% Do not edit unless you really know what you are doing.
\documentclass[11pt,american,czech]{book}
\usepackage[T1]{fontenc}
\usepackage[utf8]{inputenc}
\usepackage[a4paper]{geometry}
\geometry{verbose,tmargin=4cm,bmargin=3cm,lmargin=3cm,rmargin=2cm,headheight=0.8cm,headsep=1cm,footskip=0.5cm}
\pagestyle{headings}
\setcounter{secnumdepth}{3}
\usepackage{url}
\usepackage{amsmath}
\usepackage{amsthm}
\usepackage{amssymb}
\usepackage{graphicx}
\usepackage{setspace}
\usepackage{comment}
\usepackage{float}
\restylefloat{table}
\usepackage{amssymb}
\usepackage{booktabs}

\makeatletter
%%%%%%%%%%%%%%%%%%%%%%%%%%%%%% Textclass specific LaTeX commands.
\newenvironment{lyxlist}[1]
{\begin{list}{}
{\settowidth{\labelwidth}{#1}
 \setlength{\leftmargin}{\labelwidth}
 \addtolength{\leftmargin}{\labelsep}
 \renewcommand{\makelabel}[1]{##1\hfil}}}
{\end{list}}

%%%%%%%%%%%%%%%%%%%%%%%%%%%%%% User specified LaTeX commands.
%% Font setup: please leave the LyX font settings all set to 'default'
%% if you want to use any of these packages:

%% Use Times New Roman font for text and Belleek font for math
%% Please make sure that the 'esint' package is turned off in the
%% 'Math options' page.
\usepackage[varg]{txfonts}

%% Use Utopia text with Fourier-GUTenberg math
%\usepackage{fourier}

%% Bitstream Charter text with Math Design math
%\usepackage[charter]{mathdesign}

%%---------------------------------------------------------------------

%% Make the multiline figure/table captions indent so that the second
%% line "hangs" right below the first one.
%\usepackage[format=hang]{caption}

%% Indent even the first paragraph in each section
\usepackage{indentfirst}

%%---------------------------------------------------------------------

%% Disable page numbers in the TOC. LOF, LOT (TOC automatically
%% adds \thispagestyle{chapter} if not overriden
%\addtocontents{toc}{\protect\thispagestyle{empty}}
%\addtocontents{lof}{\protect\thispagestyle{empty}}
%\addtocontents{lot}{\protect\thispagestyle{empty}}

%% Shifts the top line of the TOC (not the title) 1cm upwards 
%% so that the whole TOC fits on 1 page. Additional page size
%% adjustment is performed at the point where the TOC
%% is inserted.
%\addtocontents{toc}{\protect\vspace{-1cm}}

%%---------------------------------------------------------------------

% completely avoid orphans (first lines of a new paragraph on the bottom of a page)
\clubpenalty=9500

% completely avoid widows (last lines of paragraph on a new page)
\widowpenalty=9500

% disable hyphenation of acronyms
\hyphenation{CDFA HARDI HiPPIES IKEM InterTrack MEGIDDO MIMD MPFA DICOM ASCLEPIOS MedInria}

%%---------------------------------------------------------------------

%% Print out all vectors in bold type instead of printing an arrow above them
\renewcommand{\vec}[1]{\boldsymbol{#1}}

% Replace standard \cite by the parenthetical variant \citep
%\renewcommand{\cite}{\citep}

\makeatother

\usepackage{babel}
\begin{document}
\def\documentdate{2. \v{c}ervence 2018}

%%\def\documentdate{\today}

\pagestyle{empty}
{\centering

\noindent %
\begin{minipage}[c]{3cm}%
\noindent \begin{center}
\includegraphics[width=3cm,height=3cm,keepaspectratio]{Images/TITLE/cvut}
\par\end{center}%
\end{minipage}%
\begin{minipage}[c]{0.6\linewidth}%
\begin{center}
\textsc{\large{}České vysoké učení technické v Praze}{\large{}}\\
{\large{}Fakulta jaderná a fyzikálně inženýrská}
\par\end{center}%
\end{minipage}%
\begin{minipage}[c]{3cm}%
\noindent \begin{center}
\includegraphics[width=3cm,height=3cm,keepaspectratio]{Images/TITLE/fjfi}
\par\end{center}%
\end{minipage}

\vspace{3cm}


\textbf{\huge{}Těžba dat z experimentů na tokamaku COMPASS}{\huge \par}

\vspace{1cm}


\selectlanguage{american}%
\textbf{\huge{}Data mining on the COMPASS tokamak experiments}{\huge \par}

\selectlanguage{czech}%
\vspace{2cm}


{\large{}Bakalářská práce}{\large \par}

}

\vfill{}

\begin{lyxlist}{MMMMMMMMM}
\begin{singlespace}
\item [{Autor:}] \textbf{Matěj Zorek}
\item [{Vedoucí~práce:}] \textbf{Ing. Vít Škvára}
\item [{Konzultant:}] \textbf{Ing. Jakub Urban, PhD}
\item [{Akademický~rok:}] 2017/2018\end{singlespace}

\end{lyxlist}
\newpage{}

~\newpage{}

~

\vfill{}


\begin{center}
- Zadání práce -
\par\end{center}

\vfill{}


~\newpage{}

~

\vfill{}


\begin{center}
- Zadání práce (zadní strana) -
\par\end{center}

\vfill{}


~\newpage{}

\noindent \emph{\Large{}Poděkování:}{\Large \par}

\noindent Chtěl bych zde poděkovat především svému školiteli ...................
za pečlivost, ochotu, vstřícnost a odborné i lidské zázemí při vedení
mé diplomové práce. Dále děkuji svému konzultantovi ................
za ................

\vfill

\noindent \emph{\Large{}Čestné prohlášení:}{\Large \par}

\noindent Prohlašuji, že jsem tuto práci vypracoval samostatně a uvedl
jsem všechnu použitou literaturu.

\bigskip{}


\noindent V Praze dne \documentdate\hfill{}Matěj Zorek

\vspace{2cm}


\newpage{}

~\newpage{}

\begin{onehalfspace}
\noindent \emph{Název práce:}

\noindent \textbf{Těžba dat z experimentů na tokamaku COMPASS}
\end{onehalfspace}

\bigskip{}


\noindent \emph{Autor:} Matěj Zorek

\bigskip{}


\noindent \emph{Obor:} Matematické inženýrstvý\bigskip{}


\noindent \emph{Zaměření:} Aplikované matematicko-stochastické metody

\bigskip{}


\noindent \emph{Druh práce:} Bakalářská práce

\bigskip{}


\noindent \emph{Vedoucí práce:} Ing. Vít Škvára, Ústav fyziky plazmatu, AV ČR
Za Slovankou 1782/3
182 00 Praha 8

\bigskip{}


\noindent \emph{Konzultant:} Ing. Jakub Urban, PhD., Ústav fyziky plazmatu, AV ČR, Za Slovankou 1782/3, 182 00 Praha 8

\bigskip{}


\noindent \emph{Abstrakt:} Abstrakt max. na 10 řádků. Abstrakt max.
na 10 řádků. Abstrakt max. na 10 řádků. Abstrakt max. na 10 řádků.
Abstrakt max. na 10 řádků. Abstrakt max. na 10 řádků. Abstrakt max.
na 10 řádků. Abstrakt max. na 10 řádků. Abstrakt max. na 10 řádků.
Abstrakt max. na 10 řádků. Abstrakt max. na 10 řádků. Abstrakt max.
na 10 řádků. Abstrakt max. na 10 řádků. Abstrakt max. na 10 řádků.
Abstrakt max. na 10 řádků. Abstrakt max. na 10 řádků. Abstrakt max.
na 10 řádků. Abstrakt max. na 10 řádků. Abstrakt max. na 10 řádků.
Abstrakt max. na 10 řádků. Abstrakt max. na 10 řádků. Abstrakt max.
na 10 řádků. Abstrakt max. na 10 řádků. Abstrakt max. na 10 řádků.
Abstrakt max. na 10 řádků. Abstrakt max. na 10 řádků. Abstrakt max.
na 10 řádků. Abstrakt max. na 10 řádků. Abstrakt max. na 10 řádků. 

\bigskip{}


\noindent \emph{Klíčová slova:} klíčová slova (nebo výrazy) seřazená
podle abecedy a oddělená čárkou

\vfill{}
~

\selectlanguage{american}%
\begin{onehalfspace}
\noindent \emph{Title:}

\noindent \textbf{Data mining on the COMPASS tokamak experiments}
\end{onehalfspace}

\bigskip{}


\noindent \emph{Author:} Matěj Zorek

\bigskip{}


\noindent \emph{Abstract:} Max. 10 lines of English abstract text.
Max. 10 lines of English abstract text. Max. 10 lines of English abstract
text. Max. 10 lines of English abstract text. Max. 10 lines of English
abstract text. Max. 10 lines of English abstract text. Max. 10 lines
of English abstract text. Max. 10 lines of English abstract text.
Max. 10 lines of English abstract text. Max. 10 lines of English abstract
text. Max. 10 lines of English abstract text. Max. 10 lines of English
abstract text. Max. 10 lines of English abstract text. Max. 10 lines
of English abstract text. Max. 10 lines of English abstract text.
Max. 10 lines of English abstract text. Max. 10 lines of English abstract
text. Max. 10 lines of English abstract text. Max. 10 lines of English
abstract text. Max. 10 lines of English abstract text. Max. 10 lines
of English abstract text. Max. 10 lines of English abstract text.
Max. 10 lines of English abstract text. Max. 10 lines of English abstract
text. Max. 10 lines of English abstract text.

\bigskip{}


\noindent \emph{Key words:} keywords in alphabetical order separated
by commas

\selectlanguage{czech}%
\newpage{}

~\newpage{}

\pagestyle{plain}

\tableofcontents{}

\newpage{}


\chapter*{Úvod}

\addcontentsline{toc}{chapter}{Úvod}

Text úvodu....

\chapter{Motivace}
\pagestyle{headings}

\section{Strojové učení}

\section{Klasifikace}

\section{Rysy (Features)}

Ve strojovém učení a rozpoznávání vzorů se pod pojmem "rys" \ (feature) rozumí individuální měřitelná vlastnost nebo charakteristika pozorovaného jevu. Výběr těchto rysů je %jedním z nejdůležitějších kroků v řešení daného problému a  
naprosto zásadní pro efektivní rozpoznávací, regresní a klasifikační algoritmy. 
Čím relevantnější  a charakterističtější rys, tím lépe jsme schopni docílít větší přesnosti modelu. Na druhou stranu vynechání zbytečných, případně méně důležitých rysů zase snižuje složitost modelu a urychluje jeho trénink.
Nejčastější forma rysu je číselná hodnota, avšak při rozpoznávání syntetických vzorků se hojně používají i písmena, slova nebo grafy.

Selekci těchto rysů je možné demonstrovat na následujícím příkladu. Předpokládejme, že bychom chtěli předvídat typ domácího mazlíčka, jež si někdo koupí.

Do rysů můžeme zahrnout například věk osoby, pohlaví, jméno, bydlení (byt, dům, ...), rodinný příjem, vzdělání a počet dětí. Je zřejmě, že většina těchto rysů nám může při předvidání pomoci, ale některé jako třeba vzdělání nebo jméno jsou zjevně méně důležité.%nedůležité. 

\begin{table}[H]
	\centering
	%\renewcommand{\arraystretch}{1}
	%\renewcommand{\tablename}{Tab.}
	
	\begin{tabular}{c|c|c|c|c|c|c}
		
		jméno & věk &	pohlaví	&	bydlení &  příjem & počet dětí & vzdělání	\\
		\hline
		Karel & 25 & muž & byt & 30.000 & 0 & středoškolské \\
		\hline
		Petr & 30 & muž & dům & 45.000 & 2 & vysokoškolské \\
		\hline
		Jana & 42 & žena & byt & 23.000 & 1 & základní \\
		\hline
		Miloš & 51 & muž & dům & 29.000 & 1 & středoškolské\\
		\hline
		\multicolumn{7}{c}{...}  \\
	\end{tabular}
	\caption{Vzorová tabulka rysů k demonstračnímu příkaldu}
\end{table}


\begin{comment}

Například pokud se snažíte předvídat typ zvířete, který si někdo zvolí, vaše vstupní funkce mohou zahrnovat věk, domácí region, rodinný příjem atd. Označení je konečnou volbou, jako je pes, ryba, iguana, rock, atd.

ve strojovém učení a rozpoznávání vzoru je rysem individuální měřitelná vlastnost nebo charakteristika pozorovaného jevu. [1] Výběr informačních, diskriminačních a nezávislých prvků je zásadním krokem pro efektivní algoritmy v rozpoznávání, klasifikaci a regresi. 

Rysy jsou nejčastěji číselné avšak při rozpoznávání syntetických vzorů se navíc používají například řetězce a grafy.  

Features are a column of data given as the input. They are also called as attributes or might sometimes be referred as dimensions.

A particular problem data set can have several features tagging to them. It is important to select the features that are more relevant to our problem so that the accuracy of the model improves. It also reduces the complexity of the model as we avoid the least significant / unnecessary feature data. The simpler model is simpler to understand and explain.

This Process is called feature engineering / selection and is one of the crucial step of pre-processing. Different algorithms can be used to implement it.

The Features can be of different types.

Simple Supervised selection where they are simple values like numbers and characters.
Eg: Size of the house (number) .

In unsupervised learning, the model is itself trained to recognize the features and work on it.

Eg: In character recognition, features may include histograms counting the number of black pixels along horizontal and vertical directions, number of internal holes, stroke detection and many others.
\end{comment}

\subsection{První a druhá derivace}
Prvním rysem použitým při  klasifikaci je první derivace.
Jelikož jsou k dispozici pouze jednotlivé body, nelze použít analytický vzorec pro derivace funkce $f(x):R \longrightarrow R$, tedy konkrétně
\begin{equation}
\frac{d f(x)}{dx}=f'(x) = \lim\limits_{h->0} \frac{f(x+h)-f(x)}{h}.
\label{derivace}
\end{equation}
Namísto toho se musí počítat numericky, a to za použití centrální diference druhého řádu pomocí \eqref{vnitřní diference} a v krajních bodech pomocí jednostranných diferencí prvního nebo druhého řádu \eqref{vnější diference}.
\begin{equation}
\hat{f}'_k = \frac{f(x_{k+1})-f(x_{k-1})}{2h} 
\label{vnitřní diference}
\end{equation}
\begin{equation}
\hat{f}'_0 = \frac{f(x_1)-f(x_0)}{h} \ \ \text{a} \ \ \hat{f}'_n = \frac{f(x_n)-f(x_{n-1})}{h}
\label{vnější diference}
\end{equation}

Dalším použitým rysem je druhá derivace, kterou lze jesnáze získat použitím víše zmíněncým vzoruců \eqref{vnitřní diference} a \eqref{vnější diference} na již jednou zderivovaný signál.


\subsection{Klouzavý průměr}
Dalším vybraným rysem je klouzavý průměr. Klouzavý průměr je diskrétní lineární filtr s konečnou dobou odezvy, který slouží k vyhlazení signálu. Nadále ho budeme značit jako $\widetilde{X}_t $. Nechť máme  vektor naměřených hodnot $\vec{X}=(x_1, x_2,...,x_n)$, pak definuji klouzavý průměr pro $t \ \in \ 1,2,...,n$ jako
\begin{equation}
\widetilde{X}_t = \frac{1}{\tilde{\omega}} \sum_{k = t-\tilde{\omega} }^{t} x_k , 
\label{klouzavý průměr}
\end{equation}  
kde $\tilde{ \omega } = min\{ \omega, t \} $, přičemž $\omega$ je délka úseku. Pak vektor $\vec{\widetilde{X}} = (\widetilde{X}_1,\widetilde{X}_2,..,\widetilde{X}_n)$ je rys vektoru $\vec{X}$.

Ve skutečnosti se jedná o standartní aritmetický průměr \eqref{aritmetický průměr}, jež je aplikovaný pouze na úsek dat konečné délky.
\begin{equation}
\overline{X}_n = \frac{1}{n} \sum_{k=1}^{n} x_k,
\label{aritmetický průměr}
\end{equation}

Mezi rysy byl vybrán, protože předpokládáme, že okamžitá hodnota je závislá na %bezprostředně 
předchozích datech. Důvod, proč využíváme jen konečně dlouhý úsek předcházejících hodnot je ten, že ze zákona velkých čísel aritmetický průměr konverguje ke střední hodnotě, a tedy ke konstantě. To znamená, že postupem času budou mít rozdílná data v odlišných stavech stejnou hodnotu rysu. Z čehož plyne, že takovýto rys by jen zkresloval a znepřesňoval výsledek, viz Obr. \ref{obr1}. 


\begin{figure} [H] 
	\renewcommand{\figurename}{Obr.}
	\centering
	\includegraphics[scale=0.7]{Obr/obr2}
	\caption{Rozdíl mezi klouzavým a aritmerickým průměrem aplikovaným na syntetická data}
	\label{obr1}
\end{figure}



\subsection{Exponenciální klouzavý průměr}
Jíž dříve jsme se zmínili, že předpokládáme závislost na předchozích hodnotách. Nicméně je zřejmé, 
%"Jak jsem se zmínil již dříve, předpoládám závislost na předchozích hodnotách. "Není však překvapením"
 že hodnoty naměřené s velkým časovým rozestupem %vzdálenější hodnoty 
na sebe mají mnohem menší vliv, než ty jež jsou naměřeny bezprostředně zasebou. Proto dalším vybraným rysem je tedy exponenciální klouzavý průměr $S_t$. 

Nechť $\vec{X}=(x_1, x_2,...,x_n)$ je vektor naměřených hodnot, pak definuji váhový součet zleva pro $t \ \in \ 1,2,...,n$ jako
\begin{equation}
S_t = \frac{1}{\tilde{\omega}} \sum_{k=t-\tilde{\omega}}^{t} g_{t-k} \cdot x_k, 
\label{exponenciální klouzavý průměr}
\end{equation} 
kde $\tilde{ \omega } = min\{ \omega, t \} $, $\omega$ je délka úseku a $g_m$ je váhová funkce tvaru $g_m = \gamma^m$, přičemž $\gamma \in (0,1)$. Pak vektor $\vec{S} = (S_1,S_2,..,S_n)$ je rys vektoru $\vec{X}$. 
Díky exponenciální váhové funkci $g_m$, jsme tedy schopni snížit důležitost více vzdálených dat, což byl náš záměr.  

\begin{comment}

\subsection{Rozptyl}
Rozptyl asi vynechám protože je k ničemu
\begin{equation}
\sigma ^2_n = \frac{1}{n} \sum_{k = 1}^{n} (x_k -E(x) )^2 = \frac{1}{n} \sum_{k = 1}^{n} (x_k - \overline{X}_n )^2
\label{rozptyl}
\end{equation}
\end{comment}


\begin{comment}

Nechť $X = (x_1, x_2, ..., x_n)$ je diskrétní náhodná veličina s příslušnými pravděpodobnostmi  $p_1,p_2, ...,p_n$. Pak střední hodnotu náhodné veličiny $X$ označím symbolem $E[X]$ a definuji ji jako 
\begin{equation}
E[X]  = \sum_{k=1}^{n} p_k \cdot x_k.
\label{etření hodnota}
\end{equation}
Pokud jsou $x_1, x_2, ..., x_n$ stejně rozdělené tzn. $p_1 = p_2 = ... = p_n$ pak stření hodnota nabývá tvaru 
\begin{equation}
E[X] = \frac{1}{n} \sum_{k=1}^{n} x_k
\label{mean}
\end{equation}
\end{comment}

\subsection{Klouzavý rozptyl}
Ve statistice a teorii pravděpodobnosti se pod pojmem rozptyl rozumí střední hodnota kvadrátu odchylky od střední hodnoty náhodné veličiny. Bývá reprezentován symbolem $Var(X)$ nebo $\sigma^2$ a definován vzorcem 
\begin{equation}
Var(X) = E[(X-E[X])^2] %= \sum_{k=1}^{n}(x_k-E[X])^2
\end{equation}
pro stejně rozdělené diskrétní náhodné veličiny $\vec{X} = (x_1, x_2, ...,x_n)$
můžeme tento vzorec přepsat do tvaru 

\begin{equation}
Var(X) = \frac{1}{n}\sum_{k=1}^{n} (x_k - \overline{X}_n)^2.
\label{varf}
\end{equation}

Bohužel, rozptyl nemůžeme jako rys použít ze stejného důvodu jako aritmetický průměr, protože s postupem času bude různým stavům přiřazovat stejnou hodnotu. 
Proto zed využijeme místo aritmetického průměru již dříve definovaný klouzavý průměr a výslednou veličinu %jsem označil jako úsekový rozptyl. Úsekový rozptyl budu nadále značit jako $D_m$ a pro náhodné veličiny $X = (x_1, x_2, ...,x_n)$ ho definuji jako
budu dále nazývat klouzavým rozptylem a značit $D_t$.

Nechť $\vec{X} = (x_1, x_2, ...,x_n)$ je vektor naměřených hodnot, pak definuji úsekový rozptyl pro $t \in 1, 2, ..., n$ jako 
\begin{equation}
D_m = \frac{1}{\tilde{\omega}} \sum_{k=t-\tilde{\omega}}^{t} (x_k - \widetilde{X}_t)^2,
\label{klouzavý rozptyl}
\end{equation}
kde $\tilde{ \omega } = min\{ \omega, t \}$, $w$ je opět délka úseku a $\widetilde{X}_t$ je klouzavý průměr \eqref{klouzavý průměr}. Pak vektor $\vec{D} = (D_1, D_2, ..., D_n)$ je rys vektoru $\vec{X}$.

\begin{figure} [H] 
	\renewcommand{\figurename}{Obr.}
	\centering
	\includegraphics[scale=0.7]{Obr/obr3}
	\caption{Rozdíl mezi úsekovým a normálním rozptylem aplikovaným na syntetická data}
	\label{obr3}
\end{figure}



\pagestyle{headings}


\chapter{Teoretická část}
\pagestyle{headings}

\section{Skrytý Markovův model}

Skrytý Markovův model známý spíše pod svým anglickým názvem "Hidden Markov model" $ \ $je statistický model, který slouží k modelování Markovských procesů se skrytýmy stavy.
Přesněji se jedná o dvojnásobně zapuštěný stochastický proces podložený dalším stochstickým procesem, který není pozorován, je tedy skrytý. Nicméně tento skrytý porces může být pozorován skrze jiné stochastické procesy, jež poskytují posloupnost pozorování. 

Skrytý Markovův model je široce používán v rozpoznávání řeči (speech recognition), modelovéní přirozeného jazyka, rozpoznávání ručněpsaného písma a analýza biologických sekvencí jako například DNA a protejnů.

%Pro lepší představu uvažujme systém uren a míčků vyobrazený na  "obrázek".
Pro lepší představu si tento model ukážeme na příkladě uren a míčků. Předpokládejme, že v místnosti je $N$ velkých skleněných uren. V každé urně je velký počet barevných míčků. Předpokládejme, že máme $M$ odlišných barev míčků. Fyzikální proces pro získání pozorování je následující. Džin je v místnosti a on (nebo ona) podle nějakého náhodného procesu vybírá počáteční urnu. Z této urny vybere náhodně míček a jeho barva je nahrána jako pozorování. Míček je pak nahrazen v urně, z níž byl vybrán. Další urna je vybrána porocesem náhodného výběru spojeného se současnou urnou a výběr míčku je opakován. Celý proces generuje konečný počet pozorování posloupnosti barev, který bychom rádi modelovaly jako pozorovaný výstup skrytého markovského modelu. (příklad převzat z \cite{Rabiner})

Dále se na tento model můžeme dívat
%Na tento model se můžeme také dívat
 jako na specifický případ stavového prostorového modelu, ve kterém jsou skryté proměnné diskrétní. 
Nicméně když prozkoumáme "jednorázový" řez modelu, vidíme že odpovídá Mixture distribution (viz kapitola $9$ v \cite{Bishop}) s hustotou pravděpodobnosti danou  $\mathbb{P}(\vec{x}|\vec{z})$.
Proto ho můžeme interpretovat jako rozšíření Mixture modelu, kde výběr složky směsi, pro každé pozorování, není nezávislý, ale závisí na volbě složek z předchozího pozorování.

Jako v případě standartního Mixture modelu, skryté proměnné jsou diskrétní multinomické proměnné $\vec{z}_n$ popisující složku směsi jež je zodpovědná za generování příslušného pozorování $\vec{x}_n$. 
Od této chvíle budeme předpokládat, že skrytý proces je Markovův, tzn. splňuje Markovu vlastnost (Markov property)\cite{Markov Chain}. Z tohoto předpokladu nyní plyne, že budoucí stav skryté proměnné $\vec{z}_n$ závisí pouze na předchozím stavu $\vec{z}_{n-1}$ skrze podmíněnou pravděpodobnost $\mathbb{P}(\vec{z}_n|\vec{z}_{n-1})$.  Pak zavedeme matici přechodů $\mathbb{A}$ danou předpisem  $A_{i,j} \equiv \mathbb{P}(\vec{z}_{n,j}|\vec{z}_{n-1,i} = 1)$, navíc matice splňuje, že $A_{i,j} \in (0,1)$ a skoupce jsou normovány na $1$, tzn $\sum_j A_{i,j} \ =1$. Dále můžeme tedy explicitně napsat podmíněné rozdělení pro $K$ skrytých stavů ve tvaru

\begin{equation}
\mathbb{P}(\vec{z}_n|\vec{z}_{n-1},\mathbb{A}) = \prod_{i=1}^{K} \prod_{j=1}^{K}
A^{\vec{z}_{n-1,i} \vec{z}_n,j}_{i,j}.
\label{podmíněná pravděpodobnost explicitní}
\end{equation}


\begin{figure} [H] 
	\renewcommand{\figurename}{Obr.}
	\centering
	\includegraphics[scale=0.3]{Obr/obr4}
	\caption{Diagram přechodů, kde skryté proměnné mají tři možné stavy odpovídající třem boxům. Šipky označují prvky matice přechodů $A_{i,j}$. \cite{Bishop}}
	\label{tří stavý diagram}
\end{figure}

Tímto vzorcem můžeme vyjádřit všechny skryté stavy až na počáteční $\vec{z}_1$. Tento stav má pouze marginální rozdělení $\mathbb{P}(\vec{z}_1)$ reprezentované vektorem pravděpodobností $\pi$, kterýž má tvar $\pi \equiv \mathbb{P}(\vec{z}_{1,j} = 1)$ a tedy 

\begin{equation}
	\mathbb{P}(\vec{z}_1|\pi) = \prod_{j=1}^{K} \pi^{\vec{z}_{1,j}}_j,
	\label{marginální pravděpodobnost}
\end{equation} 
kde $\sum_j \pi_j = 1$.
Kdybychom chtěli matici $\mathbb{A}$ vysvětlit i jinak, mohli bychom toho docílit graficky pomocí diagramu na Obr. \ref{tří stavý diagram}. Když tento diagram dále rozvyneme s průběhem času, získáme takzvaný mřížový diagram, který nám poskytuje alternativní reprezentaci přechodů mezi jednotlivými skrytými stavy. Pro případ Skritého Markovského modelu pak tento diagram nabývá tvaru Obr. \ref{HMM diagram}. 

Abychom měli pravděpodobnostní model kompletní, je třeba ještě zavést %podmíněné rozdělení pozorovaných proměnných
emisní pravděpodobnosti (emission probabilities) $\mathbb{P}(\vec{x}_n|\vec{z}_n, \Theta)$, kde $\Theta$ představuje soubor parametrů řídícího rozdělení. 
%Tyto pravděpodobnosti jsou známé jako emisní pravděpodobnosti (emission probabilities). 
Pro pozorované proměnné $\vec{x}_n$ se 
%Vzhledem k tomu, že $\vec{x}_n$ jsou pozorované,  tak se 
rozdělení $\mathbb{P}(\vec{x}_n|\vec{z}_n, \Theta)$ skládá z $K$-dimenzionálního vekoru odpovídajícího $K$ potenciálním stavům $\vec{z}_n$. %, pro dané hodnoty $\Theta$. 
Emisní pravděpodobnosti můžeme pak zapsat jako 

\begin{equation}
\mathbb{P}(\vec{x}_n|\vec{z}_n,\Theta) = \prod_{j=1}^{K} \mathbb{P}(\vec{x}_n|\Theta_j) ^{\vec{z}_n,j},
\end{equation}

přičemž mohou mít například tvar Gaussova ($R$-rozměrného) rozdělení
\begin{equation}
\mathbb{P}(\vec{x}_n|\vec{z}_n) = \prod_{j=1}^{K} \mathcal{N}(\vec{x}_n|\vec{\mu}_j,\Sigma_j) ^{\vec{z}_{n,j}} = \prod_{j=1}^{K}
\left( \frac{1}{(2\pi)^{R/2}}\frac{1}{|\Sigma_j|^{1/2}} exp \Bigg\{ -\frac{1}{2}(\vec{x}_n - \vec{\mu}_j)^T \Sigma_j^{-1}(\vec{x}_n - \vec{\mu}_j) \Bigg\} \right)^{\vec{z}_{n,j}}
\end{equation}
pakud prvky $\vec{x}_n$ jsou spojité.

Nyní se zaměříme na homogenní modely, pro které všechna podmíněná rozdělení řídící skryté proměnné sdílí stejné parametry $\mathbb{A}$ a podobně všechna emisní rozdělení sdílí stejné parametry $\Theta$.

Sdružené pravděpodobnostní rozdělení přes skryté i pozorované proměnné jsou pak dány vzorcem

\begin{equation}
\mathbb{P}(\vec{X},\vec{Z}|\widetilde{\vec{\Theta}}) = \mathbb{P}(\vec{z}_1|\pi) \prod_{n=2}^{N} \mathbb{P}(\vec{z}_n|\vec{z}_{n-1},\mathbb{A}) \prod_{m =1 }^{N} \mathbb{P}(\vec{x}_m|\vec{z}_m,\Theta),
\end{equation}
kde $\vec{X} = (\vec{x}_1,...,\vec{x}_N)$, $\vec{Z} = (\vec{z}_1,...,\vec{z}_N)$ a $\widetilde{\vec{\Theta}} =(\pi, \mathbb{A},\Theta)$ jsou parametry řídícího modelu.

\begin{figure} [H] 
	\renewcommand{\figurename}{Obr.}
	\centering
	\includegraphics[scale=0.3]{Obr/obr5}
	\caption{Stavový diagram Obr. \ref{tří stavý diagram} rozvynutý s časem. Každý sloupec diagramu odpovídá jedné skryté proměnné $\vec{z}_n$. \cite{Bishop}}
	\label{HMM diagram}
\end{figure}

\subsection{Viterbiho algoritmus}

Tento algoritmus navrhl Andrew Viterbi, již v roce 1967, za účelem dekódování konvolučních kódů, jež se používají nejen v mobilních sítích, ale také ke komunikaci se satelity a sondami ve vesmíru. V současnosti se používá k rozpoznávání a syntéze řeči, vyhledávání klíčových slov, v bioinformatice nebo, což je pro nás nejdůležitější, k hledání nejpravděpodobnějších poslopností stavů. 

V nejobecnější podobě se na Viterbiho alogritmus můžeme dívat jako na řešení problému maximálního aposteriorního pravděpodobnostního odhadu posloupnosti skrytých stavů konečného diskrétního Morkova procesu. Tento problém je formálně identický s problémem hledání nějkratší cesty grafem. Posloupnost pozorování $\vec{X}$ každé cesty může byt určena jako délka úměrná $-\ln{\mathbb{P}(\vec{Z},\vec{X})}$, kde $\vec{Z}$ je seqvence stavů spojená s příslušnou cestou. Tento poznatek nám dovoluje řešit problém hledání posloupnosti stavů pro, které je $\mathbb{P}(\vec{Z},\vec{X}) = \mathbb{P}(\vec{Z}|\vec{X}) \mathbb{P}(\vec{X})$ maximální, jako problém hledání cesty jejíž délka $-\ln{\mathbb{P}(\vec{Z},\vec{X})}$ je minimální, poněvadž $\ln{\mathbb{P}(\vec{Z},\vec{X})}$ je monotoní funkcí $\mathbb{P}(\vec{Z},\vec{X})$ a každá cesta odpovídá právě jedné posloupnosti stavů. Následně díky platnosti Markovy vlastnosti můžeme přepsat $\mathbb{P}(\vec{Z},\vec{X})$ jako 

\begin{equation}
\mathbb{P}(\vec{Z},\vec{X})  = \mathbb{P}(\vec{X}|\vec{Z}) \mathbb{P}(\vec{Z}) = \mathbb{P}(\vec{z}_1) \left[\prod_{n = 2}^{N} \mathbb{P}(\vec{z}_{n}|\vec{z}_{n-1}) \right] \prod_{n=1}^{N} \mathbb{P}(\vec{x}_n|\vec{z}_n).
\label{produkt viterbi}
\end{equation}

Po zlogaritmování \eqref{produkt viterbi} můžeme vidět, že celková délka cesty odpovídající libovolného $\vec{Z}$ je 

\begin{equation}
-\ln{\mathbb{P}(\vec{Z},\vec{X})} = \left[\sum_{n=2}^{N} -\ln{\mathbb{P}(\vec{z}_{n}|\vec{z}_{n-1})} - \ln{\mathbb{P}(\vec{x}_n|\vec{z}_n)}\right]  -\ln{\mathbb{P}(\vec{z}_1)} - \ln{\mathbb{P}(\vec{x}_1|\vec{z}_1)}, 
\label{ln cesta}
\end{equation}
kde $-\ln{\mathbb{P}(\vec{z}_{n}|\vec{z}_{n-1})} - \ln{\mathbb{P}(\vec{x}_n|\vec{z}_n)}$ je délka přechodu ze $\vec{z}_{n-1}$ do $\vec{z}_n$.


\begin{figure} [H] 
	\renewcommand{\figurename}{Obr.}
	\centering
	\includegraphics[scale=0.6]{Obr/Vi}
	\caption{(a) Mřížový diagram s délkami přechodů. M = 4, K = 5. (b) Rekurzivní hledání nejkratší cesty pomocí Viterbiho algoritmu.}
	\label{Viterbi}
\end{figure}

Ve chvíli, kdy jsme našli nejpravděpodobnější cestu a tím pádem i združené rozdělení $\mathbb{P}(\vec{Z}, \vec{X})$, potřebujeme už jen nalézt posloupnost stavů odpovídající této cestě pomocí rekurze.

Jednou z největších předností Viterbiho algoritmu je jeho efektivita. Jelikož počet možných cest roste exponenciálně s délkou řetěžce je pro většínu algouritm výpočt velice náročný a v některých případech i nemožný. To ovšem neplatí pro tento algoritmus, neboť výpočetní náročnost Viterbiho algoritmu roste pouze lineárně s délkou řetězce.

\section{K-means Clustering}
K-means clustering je typ strojového učení bez učitele, který se používá v případě, že chceme zpracovat neoznačená data, tzn. nemáme předem definované skupiny či kategorie. Právě hlavním cílem algoritmu je najít tyto skupiny.

Předpokládejme, že máme $N$ pozorování $(\vec{x}_1, ... ,\vec{x}_N)$ náhodné $R$ rozměrné veličiny $\vec{X}$ a chceme je rozdělit do nějákých $K$ shluků.  Pod pojmem shluk budeme rozumět skupinu bodů, jejichž vzájemné vzdálenosti jsou mnohem menší v porovnání se vzdálenostmi k bodům vně skupiny. Dále je pak potřeba zavést si vektory $\vec{\phi}_j$, kde $j \in 1,..,K$. Tyto vektory představují středy našich shluků a naším cílem se nyní stává nalezení množiny $\{\vec{\phi}_j \}$, tak aby bylo splněno, že součet čtverců vzdáleností každého bodu shluku k nejbližšímu vektoru $\vec{\phi}_j$ je minimální. Jinými slovy, potřebujeme minimalizovat objektovou funkci $\rho$ definovanou jako
\begin{equation}
\rho = \sum_{n=1}^{N}\sum_{j=1}^{K} b_{n,j} \ ||\vec{x}_n - \vec{\phi}_j \ ||^2,
\label{J}
\end{equation}
kde proměnné $b_{n,j} \in \{0,1\}$ příslušící každému bodu $\vec{x}_n$ indikují, zda tento bod patří $j$-tého shluku nebo nikoli. Pokud ano, $b_{n,j}$ je rovno $1$, v opačném případě nabývá $b_{n,j}$ hodnoty $0$. 
Minimálního $\rho$ lze dosáhnout pomocí dvoufázového iteračního procesu. Nejprve vybereme, nejlépe náhodně, počáteční vektory $\vec{\phi}_j$. V první fázy bereme $\vec{\phi}_j$ jako fixní a minimalizovat budeme s ohledem na $b_{n,j}$. Jelikož $\rho$ je vůči $b_{n,j}$ lineární a pro rozdílná $n$ jsou $b_{n,j}$ nezávislé, můžeme je minimalizovat pro každé $n$ zvlášť. Jednoduše bereme $b_{n,j}$ následně
\begin{equation}
 b_{n,j}  = \left\{ \begin{array}{ll}
1 &\mbox{pokud $argmin_m ||\vec{x}_n - \vec{\phi}_m||^2 = j$}\\
0 & \mbox{jinde}.\end{array} \right. 
\end{equation}
V druhé fázi je naopak $b_{n,j}$ pevné a minimalizujeme s ohledem na $\vec{\phi}_j$. Proto zderivujeme funkci $\rho$ podle $\vec{\phi}_j$ a tuto derivací položíme rovnu $0$, tzn.
\begin{equation}
2\sum_{n=1}^{N} b_{n,j}(\vec{x}_n - \vec{\phi}_j) = 0.
\label{minrho}
\end{equation}
Načež po osamostatnění $\vec{\phi}_j$. získáváme konečný tvar
\begin{equation}
\vec{\phi}_j = \frac{\sum_{n=1}^{N} b_{n,j} \vec{x}_n}{\sum_{n=1}^{N} b_{n,j}},
\label{k-mean}
\end{equation}
kde jmenovatel představuje celkový počet bodů patřících do $j$-tého shluku.
Vzorec \eqref{k-mean} lze ovšem interpretovat teké jako střední hodnotu (anglicky: mean) všech bodů příslušících do shluku $j$, odtud plyne i název "K-means".

\begin{figure} [H] 
	\renewcommand{\figurename}{Obr.}
	\centering
	\includegraphics[scale = 1.1]{Obr/kmeansViz}
	\caption{Vizualizace algoritmu K-means. Tréninková data jsou vyobrazena jako tečky a středy shluků jako křížky. Na obrázku $(a)$ jsou vykreslena původní data. V $(b)$ je možné vidět náhodně vybrané počáteční středy a $(c)-(d)$ ilustrují úvodní dvě iterace algoritmu. \cite{kmeansviz}}
	\label{K-means}
\end{figure}




\begin{comment}

 K-means algorithm. Training examples are shown as dots, and cluster centroids are shown as crosses. (a) Original dataset. (b) Random initial cluster centroids. (c-f) Illustration of running two iterations of k-means. In each iteration, we assign each training example to the closest cluster centroid (shown by "painting" the training examples the same color as the cluster centroid to which is assigned); then we move each cluster centroid to the mean of the points assigned to it.

K-means clustering is a type of unsupervised learning, which is used when you have unlabeled data (i.e., data without defined categories or groups). The goal of this algorithm is to find groups in the data, with the number of groups represented by the variable K. The algorithm works iteratively to assign each data point to one of K groups based on the features that are provided. Data points are clustered based on feature similarity. The results of the K-means clustering algorithm are:

The centroids of the K clusters, which can be used to label new data
Labels for the training data (each data point is assigned to a single cluster)
Rather than defining groups before looking at the data, clustering allows you to find and analyze the groups that have formed organically. The "Choosing K" section below describes how the number of groups can be determined.  

Each centroid of a cluster is a collection of feature values which define the resulting groups. Examining the centroid feature weights can be used to qualitatively interpret what kind of group each cluster represents.  

This introduction to the K-means clustering algorithm covers:

Common business cases where K-means is used
The steps involved in running the algorithm
A Python example using delivery fleet data
\end{comment}



\section{kNN}


\chapter{Postup práce a výsledky}
\pagestyle{headings}
\section{Způsoby vyhodnocení výsledků}
V této kapitole se budu věnovat způsobům jak ohodnotit kvalitu (přsnost) mého algoritmu.
\subsection{Přesnost}
\subsection{Správnost}
\subsection{Recall}
\subsection{F míra}

\begin{table}
	\scriptsize
	\begin{tabular}{lcrrrrrrrrrrrr}
		\toprule 
		Kombinace rysů &  délka úseku &  Accuracy &  Chyby &  F míra stavu 0 &  F míra stavu 1 &  F míra stavu 2 &  F míra průměrná &  Precision stavu 0 &  Precision stavu 1 &  Precision stavu 2 &  Recall stavu 0 &  Recall stavu 1 &  Recall stavu 2 \\
		\midrule   
		(0, 0, 1, 0, 1) &           16 &  0.866000 &    402 &        0.938991 &        0.750916 &        0.760369 &         0.816758 &           0.939512 &           0.613772 &           0.933962 &        0.938470 &        0.966981 &        0.641192 \\
		
		 (0, 0, 1, 0, 1) &           15 &  0.867333 &    398 &        0.942991 &        0.754786 &        0.748428 &         0.815401 &           0.937055 &           0.615156 &           0.952000 &        0.949002 &        0.976415 &        0.616580 \\
		 
		(0, 0, 0, 1, 1) &           18 &  0.856333 &    431 &        0.924212 &        0.734222 &        0.781065 &         0.813167 &           0.947062 &           0.589158 &           0.910345 &        0.902439 &        0.974057 &        0.683938 \\
		
		(0, 0, 1, 0, 1) &           17 &  0.862000 &    414 &        0.936087 &        0.739574 &        0.762332 &         0.812664 &           0.942664 &           0.609160 &           0.901060 &        0.929601 &        0.941038 &        0.660622 \\
		
		(0, 0, 0, 1, 1) &           16 &  0.861333 &    416 &        0.935159 &        0.736842 &        0.765565 &         0.812522 &           0.943067 &           0.592539 &           0.941399 &        0.927384 &        0.974057 &        0.645078 \\
		
		(0, 0, 0, 1, 1) &           17 &  0.858000 &    426 &        0.928632 &        0.732919 &        0.775602 &         0.812385 &           0.945434 &           0.587482 &           0.926259 &        0.912417 &        0.974057 &        0.667098 \\
		
		(0, 0, 1, 0, 1) &           14 &  0.866333 &    401 &        0.943777 &        0.760770 &        0.730924 &         0.811824 &           0.929570 &           0.622189 &           0.961945 &        0.958426 &        0.978774 &        0.589378 \\
		
		(0, 0, 0, 1, 1) &           19 &  0.853667 &    439 &        0.921670 &        0.728738 &        0.782671 &         0.811026 &           0.951594 &           0.587302 &           0.884176 &        0.893570 &        0.959906 &        0.702073 \\
		
		(0, 0, 1, 0, 1) &           18 &  0.859000 &    423 &        0.932884 &        0.735675 &        0.763449 &         0.810669 &           0.945361 &           0.604863 &           0.885470 &        0.920732 &        0.938679 &        0.670984 \\
		
		(0, 0, 0, 1, 1) &           15 &  0.859333 &    422 &        0.937743 &        0.732270 &        0.750392 &         0.806802 &           0.940357 &           0.586648 &           0.952191 &        0.935144 &        0.974057 &        0.619171 \\
		\bottomrule
	\end{tabular}
\end{table}

\begin{table}
	\scriptsize
	\begin{tabular}{lllllllllll}
		\toprule 
		&                0 &                1 &                2 &                3 &                4 &                5 &                6 &                7 &                8 &                9 \\
		\midrule
		Kombinace rysů    &  (0, 0, 1, 0, 1) &  (0, 0, 1, 0, 1) &  (0, 0, 0, 1, 1) &  (0, 0, 1, 0, 1) &  (0, 0, 0, 1, 1) &  (0, 0, 0, 1, 1) &  (0, 0, 1, 0, 1) &  (0, 0, 0, 1, 1) &  (0, 0, 1, 0, 1) &  (0, 0, 0, 1, 1) \\
		délka úseku       &               16 &               15 &               18 &               17 &               16 &               17 &               14 &               19 &               18 &               15 \\
		Accuracy          &            0.866 &         0.867333 &         0.856333 &            0.862 &         0.861333 &            0.858 &         0.866333 &         0.853667 &            0.859 &         0.859333 \\
		Chyby             &              402 &              398 &              431 &              414 &              416 &              426 &              401 &              439 &              423 &              422 \\
		F míra stavu 0    &         0.938991 &         0.942991 &         0.924212 &         0.936087 &         0.935159 &         0.928632 &         0.943777 &          0.92167 &         0.932884 &         0.937743 \\
		F míra stavu 1    &         0.750916 &         0.754786 &         0.734222 &         0.739574 &         0.736842 &         0.732919 &          0.76077 &         0.728738 &         0.735675 &          0.73227 \\
		F míra stavu 2    &         0.760369 &         0.748428 &         0.781065 &         0.762332 &         0.765565 &         0.775602 &         0.730924 &         0.782671 &         0.763449 &         0.750392 \\
		F míra průměrná   &         0.816758 &         0.815401 &         0.813167 &         0.812664 &         0.812522 &         0.812385 &         0.811824 &         0.811026 &         0.810669 &         0.806802 \\
		Precision stavu 0 &         0.939512 &         0.937055 &         0.947062 &         0.942664 &         0.943067 &         0.945434 &          0.92957 &         0.951594 &         0.945361 &         0.940357 \\
		Precision stavu 1 &         0.613772 &         0.615156 &         0.589158 &          0.60916 &         0.592539 &         0.587482 &         0.622189 &         0.587302 &         0.604863 &         0.586648 \\
		Precision stavu 2 &         0.933962 &            0.952 &         0.910345 &          0.90106 &         0.941399 &         0.926259 &         0.961945 &         0.884176 &          0.88547 &         0.952191 \\
		Recall stavu 0    &          0.93847 &         0.949002 &         0.902439 &         0.929601 &         0.927384 &         0.912417 &         0.958426 &          0.89357 &         0.920732 &         0.935144 \\
		Recall stavu 1    &         0.966981 &         0.976415 &         0.974057 &         0.941038 &         0.974057 &         0.974057 &         0.978774 &         0.959906 &         0.938679 &         0.974057 \\
		Recall stavu 2    &         0.641192 &          0.61658 &         0.683938 &         0.660622 &         0.645078 &         0.667098 &         0.589378 &         0.702073 &         0.670984 &         0.619171 \\
		\bottomrule
	\end{tabular}
\end{table}

\chapter*{Závěr}

\pagestyle{plain}

\addcontentsline{toc}{chapter}{Závěr}

Text závěru....
\begin{thebibliography}{1}
\bibitem{Bishop}C. M. Bishop, Pattern recognition and machine learning. Springer, New York, 2013.

\bibitem{Kikuchi}M. Kikuchi,  K. Lackner, M. Q. Tran, Fusion physics. International Atomic Energy Agency, Vienna, 2012.

\bibitem{Rabiner}L. R. Rabiner, A tutorial on hidden Markov models and selected applications in speech recognition. Proceedings of the IEEE 77(2), 1989, 257-286.

\bibitem{Markov Chain}N. Privault, Understanding Markov Chains: Examples and Applications. Springer, 2013.

\bibitem{Viterbi}G. D. Forney, The Viterbi Algorithm. Proceedings of the IEEE 61(3) 1973, 268-278.

\bibitem{Machine learning in action}P. Harrington, Machine Learning in Action. Minning Publications Co. Greenwich, 2012.

\bibitem{kmeansviz}{http://stanford.edu/~cpiech/cs221/img/kmeansViz.png}

\end{thebibliography}

\end{document}
